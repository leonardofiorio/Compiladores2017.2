
\documentclass{beamer}

\usetheme{Szeged}


\title{Projeto Compilador para Linguagem Lua}

\author{Fernando da Ros e Leonardo Fiorio Soares}

\institute[Universidade Federal Fluminense] 

\date{2017}

\begin{document}

\begin{frame}
  \titlepage
\end{frame}



\begin{frame}
        \frametitle{Hist\'oria da Linguagem Lua}
        
        Lua foi criada por um grupo de desenvolvedores da PUC-Rio com objetivo de ser usada pela Petrobrás. Suas qualidades atraíram outros programados que aplicaram-na em outros projetos. Hoje, a linguagem Lua é desenvolvida no laboratório LabLua do Departamento de Informática da pr\'opria PUC-Rio.
\end{frame}




\begin{frame}
    \frametitle{Caracter\'isticas da Linguagem Lua}
    
    \begin{itemize}
    \item Multiparadigma
    
         \begin{itemize}
    
            \item Procedural
            \item Orientada a Objetos
            \item Orientada a Dados
            \item Funcional 
            \item Orientada a Descrição de Dados            
    
         \end{itemize}
         
    \end{itemize}
    
    \begin{itemize}
             \item Projetada para estender aplica\c{c}\~oes
                \begin{itemize}
                    \item Sem\^antica extens\'ivel
                    \item Sua API se comunica com outras linguagens como C, C++, Java, C#, Smalltalk, Fortran...
                \end{itemize}        
    
        
     \end{itemize}
    
\end{frame}

\begin{frame}
    \frametitle{Caracter\'isticas da Linguagem Lua}
    
    \begin{itemize}
        \item Lua oferece meta-mecanismos para construções na linguagem.
        \begin{itemize} Exemplo de orientação a objetos:
                    \item Classes
                    \item Herança
        \end{itemize}
            
      
    \end{itemize}
    
   
    
\end{frame}

\begin{frame}
    \frametitle{Utiliza\c{c}\~ao da Linguagem Lua}
   
    
\end{frame}
    

\end{document}


