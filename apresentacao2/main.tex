
\documentclass{beamer}
\usepackage[utf8]{inputenc}
\usepackage[portuguese]{babel}

\usetheme{Szeged}


\title{Projeto Compiladores - Linguagem Lua}

\author{Fernando Da Rós e Leonardo Fiório Soares}

\institute[Universidade Federal Fluminense] 

\date{2017}

\begin{document}

\begin{frame}
  \titlepage
\end{frame}

\begin{frame}
        \frametitle{História da Linguagem Lua}
        Lua foi criada por um grupo de desenvolvedores da PUC-Rio com objetivo inicial de ser usada pela Petrobras. Devido à sua eficiência, clareza e facilidade de aprendizado, passou a ser usada em diversos ramos da programação. Hoje, a linguagem Lua é desenvolvida no laboratório LabLua do Departamento de Informática da própria PUC-Rio.

        \centering
        \includegraphics[width=3cm]{images/lua.png}
\end{frame}

\begin{frame}
    \frametitle{Características da Linguagem Lua}
    
    \begin{itemize}
    \item Multiparadigma
    
         \begin{itemize}
            \item Procedural
            \item Orientada a Objetos
            \item Orientada a Dados
            \item Orientada a Descrição de Dados            
            \item \textbf{Funcional }
         \end{itemize}
    \end{itemize}
    \begin{itemize}
             \item Projetada para estender aplicações
                \begin{itemize}
                    \item Semântica extensível
                    \item Sua API se comunica com outras linguagens como C, C++, Java, C#, Smalltalk, Fortran...
                \end{itemize}        
     \end{itemize}
\end{frame}

\begin{frame}
    \frametitle{Características da Linguagem Lua}
    \begin{itemize}
        \item Lua oferece meta-mecanismos para construções na linguagem que sejam mais específicas. Isso ocorre sem que suas características básicas se percam.
        \item Simplicidade
        \item Portabilidade
            \begin{itemize}
                \item Para todas plataformas com compilador C
            \end{itemize}    
        \item Linguagem livre e de código aberto
    \end{itemize}
\end{frame}

\begin{frame}
    \frametitle{Características da Linguagem Lua}
    \begin{itemize}
        \item Tipagem Dinâmica: tipo da variável é definido quando o conteúdo é atribuído
         \item Favorece redigibilidade
         \item Pode dificultar procura por erros simples cometidos pelo programador
    \end{itemize}
\end{frame}

\begin{frame}
    \frametitle{Utilização da Linguagem Lua}
    \begin{itemize}
        \item Amplamente utilizada, principalmente na indústria de jogos, mas está presente também em outros softwares importantes e muito usados.
        \begin{columns}
            \begin{column}{5cm}
                \includegraphics[width=5cm]{images/wow.jpg}
            \end{column}
            \begin{column}{4cm}
                \includegraphics[width=3.5cm]{images/vlc.jpg}
            \end{column}
            \begin{column}{4cm}
                \includegraphics[width=2.5cm]{images/mysql_workbench.png}
            \end{column}
        \end{columns}
    \end{itemize}
\end{frame}

\begin{frame}
    \frametitle{Processo de Compilação da Linguagem Lua}
        Linguagem interpretada\\
        
        Lua tem seu código fonte compilado gerando um bytecode que será interpretado pela máquina virtual baseada em registradores, garantindo sua portabilidade.
    
\end{frame}

\begin{frame}
    \frametitle{Estrutura da Linguagem Lua}
        Tipos 
            \begin{itemize}
                \item nil
                \item boolean
                \item string
                \item number (sempre ponto flutuante)
                \item function
                \item table
                \item thread
                \item userdata
            \end{itemize} 
\end{frame}


\begin{frame}
    \frametitle{Estrutura da Linguagem Lua}
     \begin{itemize}
        Palavras reservadas 
        \begin{columns}
            \begin{itemize}
            \begin{column}{5cm}
                \item and
                \item end
                \item in
                \item repeat
                \item break
                \item false
                \item local
                \item return
                \item do
                \item for
                \item nil
            \end{column}
            \begin{column}{5cm}
                \item then
                \item else
                \item function
                \item not
                \item true
                \item elseif
                \item if
                \item or
                \item until
                \item while
            \end{column}
            \end{itemize} 
        \end{columns}
    \end{itemize}
\end{frame}


\begin{frame}
    \frametitle{Exemplo de código}
    if a==1 then\\
        \begin{tabular} /print("variavel a é igual a 1")
        \end{tabular}\\
    end\\
    \newline
    cont = 1\\
    while cont <= 10 do\\
        \begin{tabular} /print(cont)\\
        cont = cont + 1
        \end{tabular}\\
    end
\end{frame}

\begin{frame}
    \frametitle{Vantagens da escolha}
    \begin{itemize} 
        \item Vantagens
        \begin{itemize}
            \item Simplicidade 
            \item Linguagem leve
            \item Existência de bibliotecas especializadas em pattern-matching
                \begin{itemize}
                    \item LPEG
                \end{itemize}
        \end{itemize}
        \item Desvantagens
        \begin{itemize}
            \item Possível falta de ferramentas específicas para de desenvolvimento de compiladores que auxiliem a programação
            \item Comparada a outras linguagens, pode aumentar o tempo necessário na busca de soluções para eventuais problemas devido ao menor número de comunidades
        \end{itemize}
    \end{itemize}
\end{frame}

\begin{frame}
    \frametitle{Vantagens da escolha - Exemplo Utilização LPEG}
    \includegraphics[width=10cm]{images/lpeg.png}
\end{frame}

\begin{frame}
    \frametitle{Principais comandos da biblioteca LPEG}
    \includegraphics[width=10cm]{images/lpeg-commands.png}
\end{frame}

\begin{frame}
    \frametitle{Vantagens da escolha - Exemplo Utilização LPEG}
    \includegraphics[width=10cm]{images/lpeg2.png}
    \begin{itemize}
        \item A biblioteca LPEG é uma ferramenta poderosa que permite sua utilização em gramáticas diferentes. 
        \item O exemplo acima ilustra a implementação de uma função que tem o objetivo de separar "pedaços" da expressão, a partir de um separador passado como parâmetro e uma expressão. 
    \end{itemize}
\end{frame}

\begin{frame}
    \frametitle{Fatorial}
    \includegraphics[width=5cm]{images/ast_fatorial.png}
    \centering\includegraphics[width=5cm]{images/fatorial_normal.jpg}
\end{frame}

\begin{frame}
    \titlepage{ Segunda Apresentação}
\end{frame}

\begin{frame}
    \frametitle{Implementação de expressões e comandos}
    \begin{itemize}
        \item As estruturas de dados definidas foram:
        \begin{itemize}
            \item Pilha S
            \item Tabela M
            \item Pilha C
            \item Árvore (AST) 
            \begin{itemize}
                \item Possui tamanho variável, ou seja, é constituída por nós que tem uma lista de filhos
                \item Seus nós são do tipo node("data", \{ children \})
            \end{itemize}
            \item Tabela E
        \end{itemize}
        \item As funções resolverExpressoes e resolverComandos são as responsáveis pela recursão sobre a árvore e executando o SMC
    \end{itemize}
\end{frame}

\begin{frame}
    \frametitle{Memória (Antes)}
    Antes nossa memória era constituída de pares de identificadores e valores, ou seja, para cada identificador tinhamos um valor associado.
    
\end{frame}

\begin{frame}
    \frametitle{Memória (Agora)}
    Nossa estrutura de memória foi adaptada para suportar o conceito de ambiente. Agora temos uma tabela memória M que guarda pares loc - valor, e outra tabela ambiente E que guarda pares identificador - loc com um ponteiro para o valor da variável na tabela M ou, no caso das constantes, guarda diretamente seu valor.
    \newline
    \newline
    Lembrando que loc foi modelado como uma classe (internamente uma tabela para lua) com atributo value armazenando o conteúdo da variável.
    
\end{frame}

\begin{frame}
    \frametitle{Diferença variável e constante}
    A função isLoc da classe Loc retorna true para objetos do tipo Loc e false caso contrário. Isso possibilita que sejam diferenciados objetos armazenados no ambiente para variável ou constante
    \newline
    Internamente, o método verifica através de type(obj) se o retorno será "table". Se for o caso, o objeto passado como parâmetro é um objeto do tipo Loc.
    
\end{frame}

\begin{frame}
    \frametitle{Parser}
    Para o parser utilizamos como base o LuaFish("Parses Lua to abstract syntax tree (AST) using LPeg") e fizemos algumas adaptações para interpretar a linguagem L.
    \newline
    \newline
    \newline
    \newline
    \newline
    \url{https://github.com/davidm/lua-fish}
\end{frame}

\begin{frame}
    \frametitle{LuaFish}
    Definindo elementos básicos da gramatica:\newline
    \includegraphics[width=10cm]{images/gramma_example.png}
    
\end{frame}

\begin{frame}
    \frametitle{LuaFish}
    Inicializando a gramatica:\newline
    \includegraphics[width=5cm]{images/init_grammar.png}
    \newline
    \newline
    \includegraphics[width=10cm]{images/chunck.png}
    \newline
    
\end{frame}

\begin{frame}
    \frametitle{LuaFish}
    Definindo as regras da gramatica:\newline
    \includegraphics[width=10cm]{images/grammar_rule.png}
    \newline
    
\end{frame}


\begin{frame}{Executando o projeto}
    Na pasta do projeto:
    \begin{itemize} 
        \item export LUA\_PATH='parser /lib/?.lua;parser/examples/?.lua;?.lua'
        \item lua parser/bin/luafish.lua "var a = 0; if 2==3 then a=1; a=2; a=3; a=5 else a=10 end; a=a+1"
        \item lua parser/bin/luafish.lua "var a =0 if a==0 then a = 1 else a=2 end"
        \item lua parser/bin/luafish.lua "var a =0; a=3;"
        \item lua parser/bin/luafish.lua "const a =0; a=3;" (Erro)
        \item lua parser/bin/luafish.lua "if 2==2 then var a=0 else var b=1 end a=3 "
    \end{itemize}
\end{frame}

\begin{frame}
    \frametitle{Referências}
    \begin{itemize}
        \item CELES, Waldemar. FIGUEIREDO, Luiz Henrique de. IERUSALIMSCHY, Roberto. A Linguagem Lua e suas Aplicações em Jogos. Disponível em https://www.lua.org/doc/wjogos04.pdf\  Acesso em 26 de Agosto de 2017.
        \item Página Oficial Linguagem Lua, A Linguagem de Programação Lua Disponível em https://www.lua.org/portugues.html. Acesso em 26 de Agosto de 2017.
        \item LEAL Marcus, Lua. Disponível em http://www.inf.puc-rio.br/~inf1621/lua1.pdf. Acesso em 26 de Agosto de 2017
    \end{itemize}
\end{frame}

\end{document}



